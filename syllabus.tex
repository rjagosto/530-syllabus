% Remember to use -shell-escape to compile this document for example 
% latexmk  -pdf -shell-escape syllabus.tex
\documentclass[10pt]{article}
\usepackage{bibentry,natbib,url,comment,amsmath}
\usepackage{graphicx,svn,parskip}
\usepackage{sometexdefs}

%\usepackage[spanish]{babel}
\usepackage[T1]{fontenc}
\usepackage{textcomp}
\usepackage[utf8]{inputenc}
\usepackage{tgtermes}
%\usepackage{tgpagella}
%\usepackage[sc]{mathpazo}

\renewcommand*{\bfdefault}{bx}
\renewcommand{\oldstylenums}[1]{%
	{\fontfamily{pplj}\selectfont #1}}
\usepackage{microtype}

\hyphenation{Incomplete}

\bibliographystyle{apalike}
\usepackage[letterpaper,bottom=.75in,top=1in,right=1in,lmargin=1.15in]{geometry}

\usepackage{advdate}

\usepackage[compact,nobottomtitles*]{titlesec} %nobottomtitles
\titleformat{\part}[hang]{\large\scshape}{\hspace{-.75in}\thepart}{.5em}{}{}
\titleformat{\section}[hang]{\large\bfseries}{\hspace{-.75in}\thesection}{.5em}{}{}
\titleformat{\subsection}[leftmargin]{\small\bfseries\filleft}{\thesubsection}{.5em}{\hspace{-.75in}}{}
\titleformat{\subsubsection}[leftmargin]{\itshape\filleft}{\thesubsubsection}{.2em}{\hspace{-.75in}}{}
\titleformat{\paragraph}[runin]{\bfseries}{\theparagraph}{0em}{}{}

\titlespacing{\part}{0ex}{.5ex plus .1ex minus .2ex}{-.25\parskip}
\titlespacing{\section}{0ex}{1.5ex plus .1ex minus .2ex}{-.25\parskip}
\titlespacing{\subsection}{0ex}{.5ex plus .1ex minus .1ex}{1ex}
\titlespacing{\subsubsection}{0ex}{.5ex plus .1ex minus .1ex}{1ex}
\titlespacing{\paragraph}{0em}{1ex}{.5ex plus .1ex minus .1ex}

\newenvironment{introstuff} {\setcounter{secnumdepth}{0}%
	\titlespacing*{\section}{-.75in}{1em}{0em}{}%
	\titleformat{\subsection}[leftmargin]{\itshape\filleft}{\thesubsection}{.5em}{\hspace{-.75in}}{}%
	\titleformat{\subsubsection}[leftmargin]{\itshape\filleft}{\thesubsubsection}{.2em}{\hspace{-.75in}}{}%
	\titleformat{\paragraph}[hang]{\bfseries}{\theparagraph}{0em}{}{}%
	\titlespacing{\subsection}{2pc}{1.5ex plus .1ex minus .2ex}{1pc}%
	\titlespacing{\paragraph}{0em}{1ex}{0ex plus .1ex minus .1ex}%
	\titlespacing{\subsubsection}{2pc}{1.5ex plus .1ex minus .2ex}{1pc}%
} {\setcounter{secnumdepth}{0}%
	\titleformat{\part}[hang]{\large\bfseries}{\hspace{-.75in}\thepart}{.5em}{}{}%
	\titleformat{\section}[hang]{\large\bfseries}{\hspace{-.75in}\thesection}{.5em}{}{}%
	\titleformat{\subsection}[leftmargin]{\small\bfseries\filleft}{\thesubsection}{.5em}{\hspace{-.75in}}{}%
	\titleformat{\subsubsection}[leftmargin]{\itshape\filleft}{\thesubsubsection}{.2em}{\hspace{-.75in}}{}%
	\titleformat{\paragraph}[runin]{\bfseries}{\theparagraph}{0em}{}{}%
	% \titleformat{\subsection}[hang]{\itshape}{\thesubsection}{.5em}{}{}%
	\titlespacing{\section}{2pc}{1.5ex plus .1ex minus .2ex}{1pc}%
	\titlespacing{\paragraph}{0em}{1ex}{0ex plus .1ex minus .1ex}%
	\titlespacing{\subsubsection}{2pc}{1.5ex plus .1ex minus .2ex}{1pc}%
}

% Create new title appearance
\makeatletter
\def\maketitle{%
	%\null
	\thispagestyle{empty}%
	\begin{center}\leavevmode
		\normalfont
		{\large \bfseries\@title\par}%
		{\large \@author\par}%
		{\large \@date\par}%
	\end{center}%
	\null }
\makeatother

\usepackage{fancyhdr}
% \renewcommand{\sectionmark}[1]{\markright{#1}{}}

\fancypagestyle{myfancy}{%
	\fancyhf{}
	% \fancyhead[R]{\small{Page~\thepage}}
	\fancyhead[R]{\small{Intro. Applied Stats -- Fall 2016 -- \thepage}}
	\fancyfoot[R]{\footnotesize{Version~of~\input{|"date"}}}
	% \fancyfoot[R]{\small{\today -- Jake Bowers}}
	\renewcommand{\headrulewidth}{0pt}
	\renewcommand{\footrulewidth}{0pt}}

\pagestyle{myfancy}

\newcommand{\entrylabel}[1]{\mbox{\textsf{#1:}}\hfil}

%% These next lines tell latex that it is ok to have a single graphic
%% taking up most of a page, and they also decrease the space arou
%% figures and tables.
\renewcommand\floatpagefraction{.9} \renewcommand\topfraction{.9}
\renewcommand\bottomfraction{.9} \renewcommand\textfraction{.1}
\setcounter{totalnumber}{50} \setcounter{topnumber}{50}
\setcounter{bottomnumber}{50} \setlength{\intextsep}{2ex}
\setlength{\floatsep}{2ex} \setlength{\textfloatsep}{2ex}

\specialcomment{com} {\begingroup\sffamily\small\bfseries}{\endgroup}
\excludecomment{com}

\title{Introduction to Applied Statistics in Political Science: \\
	Description, Comparison, Estimation, Inference}


\author{Jake Bowers \\
	\small{jwbowers@illinois.edu \\
		Online:
		\url{http://jakebowers.org/}}
}

\date{Fall 2016}

%\usepackage[pdftex,colorlinks=TRUE,citecolor=blue]{hyperref}
\usepackage[colorlinks=TRUE,citecolor=blue]{hyperref}

\renewcommand{\bibname}{ }
% \renewcommand{\refname}{\normalsize{Required:}}
\renewcommand{\refname}{\vspace{-2em}}

\def\themonth{\ifcase\month\or
	January\or February\or March\or April\or May\or June\or
	July\or August\or September\or October\or November\or December\fi}

\begin{document}
\maketitle

\begin{introstuff}


	\section{Overview}

	\subsection{When/Where}

	We meet 3:30 to 5:50pm in 404 David Kinley Hall. Moodle
	\url{https://learn.illinois.edu/course/view.php?id=19118}

	\subsection{Office Hours}

  Jake will have office hours 1:30--2:30 Mondays and Thursdays. Please make an
  appointment if you want to come to office hours to ensure that we can meet
  and talk.

	\subsection{Introduction}

	This is the start of your practice with the skills and concepts that are
	commonly used by political scientists to make arguments about how
	observation can teach us about specific implications of theory. This is a
	practice because the field constantly changes and because the subject is so
	deep and important that no one can ever truly master it. We are all always
	learning. So this course is to help you start the learning that you will
	continue for your whole life as an academic.

	As a practice, statistics involves skills and concepts. If you don't have
	the skills, then the concepts are not concrete and are difficult to
	understand. In most statistics PhD programs the skills involve the
	mathematics of linear algebra and calculus and mathematical problem solving
	skills involved in deductive proofs and algebraic manipulation. In this
	course, we are going to use a flexible computer programming system in lieu
	of math in order to demonstrate to ourselves and make concrete the concepts
	that we must internalize and use in order to apply statistics to help us
	learn about the world and about our theories. (Math, after all, is a
	language. R, the language we will be using, is another way to express
	abstract ideas.)  A by product of using R to engage with statistical concepts is that
	you'll also practice how to use R to solve data problems. That is, you will
	start to practice some of the basics of "data science" on the way to
	practice some of the basics of statistics (which is the discipline devoted
	to learning how to do science).

	This course is also a graduate course in applied statistics or political
	methodology. This means that it exists within a series of continuous
	developments within multiple disciplines. The contents of this course will
	change over time because the disciplines change over time. What we thought
	worked well in the past, may not be what we think works well today. What we
	teach in this course today may be seen by future scholars (we hope) as old
	fashioned and suboptimal. That is, this course is just like any other phd
	level graduate class: we engage with the past in order to do things
	differently in the future. This class does not aim, therefore, to teach you
	to do political science as it was done in the 1950s, 1960s, 1970s, or even
	last year. It is a tradition-based but future oriented class just like all
	of your other classes.

	\section{Goals and Expectations}
	This class aims to help you learn to think about what it means to do
	statistical inference for both descriptive and causal claims.

	The point of the course is to position you to do the future learning
	that is at the core of your work as an academic analyzing data.

	I also hope that this course will help you continue to develop the
	acumen as a reader, writer, programmer and social scientist essential
	for your daily life as a social science researcher.

	\subsection{Expectations}
	First and foremost, I assume you are eager to learn. Eagerness,
	curiosity and excitement will impel your energetic engagement with the
	class throughout the term. If you are bored, not curious, or unhappy
	about the class you should come and talk with me
	immediately. Energetic engagement manifests itself in meeting with
	your classmates outside of the class, in asking questions during the
	class, and in taking the term paper seriously.

	Second, I assume you are ready to work. Learning requires work. As much as
	possible I will encourage you to link practice directly to application
	rather than merely as a opportunity for me to rank you among your peers.
	Making work about learning rather than ranking, however, will make our work
	that much more difficult and time consuming. You will make errors. These
	errors are opportunities for you to learn --- some of your learning will be
	about how to help yourself and some will be about statistics. If you have
	too much to do this term consider dropping the course. Graduate
	school is a place for you to develop and begin to pursue your own
	intellectual agendas: this course may be important for you this term, or it
	may not. That is up for you to decide.

	Third, I assume you are willing to go along with my decisions about the
	material and sequence. I will be open to constructive and concrete
	suggestions about how to teach the class as we go along, and I will value
	such evaluations at any point in the class. I have made changes to this
	course in the middle of the term upon hearing great and useful ideas from
	students. I am happy to do so. That said, if you do not think you need to
	take this course, then don't take it.

	Fourth, I assume some previous engagement with high school mathematics.

	\subsection{Rules}

	There aren't many rules for the course, but they're all important. First,
	ask questions when you don't understand things; chances are you're not
	alone.  Second, don't miss class or section. Third, do the work. This
	does not mean divide the work up among your classmates so that you only do
	part of the work. Each person should engage with all of the work even if the
	people who writes it up changes from week to week.

	All papers written in this class will assume familiarity with the
	principles of good writing in \citet{Beck:1986}.

	All final written work will be turned in as pdf files unless we have another
	specific arrangement.\footnote{For example, if you have some reason why pdf
		files make your life especially difficult, then of course I will work with
		you find another format.} I will not accept Microsoft, Apple, or any other
	proprietary format.

	\subsection{Late Work}
	I do not like evaluation for the sake of evaluation. Evaluation should
	provide opportunities for learning.
	If you think that you and/or the rest of the
	class have a compelling reason to change the due date on one of those
	assignments, let me know in advance and I will probably just change
	the due date for the whole class.

	\subsection{Incompletes}
	Incompletes are fine in theory but terrible in practice. I urge you to
	avoid an incomplete in this class. If you must take an incomplete, you
	must give me \emph{at least} 2 months from the time of turning in an
	incomplete before you can expect a grade from me. This means that if
	your fellowship, immigration status, or job depends on erasing an
	incomplete in this class, you should not leave this incomplete until
	the last minute.

	\subsection{Participation}
	We will be doing hands-on work nearly every class
	meeting. I will lecture very little and instead will use the assignments
	given out the previous week to raise questions about  statistical theory,
	research design, and data, which will require us to confront and apply the
	reading that prepared us for the day's work. I will break away to
	draw on the board or demonstrate on my own computer now and then if everyone
	is running into the same problem or is asking a question that raised by the
	work.

	\subsection{Explorations}

	Every week or so, I will ask you to complete a short assignment that
	encourages you to engage creatively with the topics of interest. I
	anticipate that you will work on these assignments in groups and that each
	of you will come to class prepared to discuss them. I don't think that the
	groups should have more than 3 people in them. However, I'm willing to have
	larger groups if you talk with me about it.

	You will need to turn it in the day before class at 5pm so that I can read
	it prepare for class.

	\subsection{Daily R}

	Five days per week, you will need to practice writing R code. I will require
	that you make a Github Gist
	\url{https://help.github.com/articles/about-gists/} each day written in either R or
	R markdown format in which you load a dataset from the web, and learn
	something of interest to you about the units represented by those data. I'm
	imagining 2 to 10 lines of R code. Then you will paste the url to that gist
	into a Moodle Journal or a Google Spreadsheet (not sure which yet). Each
	gist must be written so that it runs from start to finish on any computer
	--- not just yours. 

	Each week I will choose at random two gists to discuss in class.

	\subsection{Grades are Feedback}

	Humans need feedback to close the intention to action gap. They also need
	feedback to feel good about their progress and to motivate them. In this
	class I will use grades as feedback. All grades except for the final grade
	will be satisfactory, unsatisfactory (with the possibility of
	"outstanding"), and fail. These map roughly onto A=satisfactory,
	C=unsatisfactory, and F=fail (i.e. you didn't try).

	I'll calculate your grade for the course this way: 20\% daily R (you have 5
	days out of every 7 to turn it in, no late work accepted, satisfactory if
	you turned it in, fail if you didn't turn it in); 10\% daily R in class
	discussion (satisfactory if the gist runs from start to finish on my
	computer or on a randomly chosen classmate's computer, unsatisfactory
	otherwise, fail if there is no gist from you when your gist is chosen, no
	late work accepted); 30\% explorations (everyone in the group receives the
	same grade, satisfactory if you are creative and thoughtful and diligent,
	unsatisfactory if you are not or if you don't seem to be getting the
	concepts, no late work accepted); 30\% in class participation (satisfactory
	if you ask good questions that show that you have thought about the
	material, a good question can be a simple question; unsatisfactory if your
	questions show that you are not doing the reading and/or are not actively
	involved in the explorations; fail if you are not in class); 10\% attendance
	(satisfactory if you show up, fail if not).

	You can miss two classes without grade penalty although I suspect you'll be
	sad to miss the discussion.

	Because moments of evaluation are also moments of learning in this
	class, I do not curve. If you all perform at 100\%, then I will give
	you all As.

	\subsubsection{Involvement}

	Quality class participation does not mean ``talking a lot.''  It includes
	coming to class; turning in assignments on time; thinking and caring about
	the material and expressing your thoughts respectfully and succinctly in
	class.  As much as possible, we will be working in groups. This work will
	require that you come prepared to meetings with your classmates and also to
	the class itself and that you are an active collaborator. Involvement also
	means meeting with your classmates at least once per week outside of class
	to complete the work begun in the class.

	\textbf{This class is an opportunity to practice courage:} I expect you to
	make a guess when I ask a question (in writing or in person), I expect that
	you will ask a question when you have a problem.

	\subsection{Computing}
	We will be using R~in class so those of you with laptops available
	should bring them. Of course, I will not tolerate the use of computers
	for anything other than class related work during active class
	time. Please install R~(\url{http://www.r-project.org}) on your
	computers before the first class session.

	Computing is an essential part of modern statistical data analysis --- both
	for turning data into information and for conveying that information
	persuasively (and thus transparently and reliably) to the scholarly
	community. In this course we will pay attention to computing, with special
	emphasis on understanding what is going on behind the scenes. You will be
	writing your own routines for a few simple and common procedures: your own
	likelihood functions, your own least squares solvers, your own bootstrapping
	and permutation statistical inference routines, and your own posterior
	distribution for Bayesian statistical inference.

	Most applied researchers use two or three computing packages at any
	one time because no single language or environment for statistical
	computing can do it all. In this class, I will be using the R
	statistical language.  You are free to use other languages, although I
	suspect you will find it easier to learn R~unless you are
	already a code ninja in some other language that allows matrix
	manipulation, optimization, and looping.

	As you work on your papers, you will also learn to write about data
	analysis in a way that sounds and looks professional by using either R
	markdown or Sweave (R+\LaTeX). No paper will be
	accepted without a code appendix or reproduction archive attached (or
	available to me online). No paper will be accepted unless it is in
	Portable Document Format
	(\href{http://en.wikipedia.org/wiki/Portable_Document_Format}{pdf}).\footnote{Actually,
		I'm willing to consider HTML or Postscript although practice with
		pdf will help you most in submitting papers to journals and other
		forms of scholarly communication.} No paper will be accepted with cut
	and pasted computer output in the place of well presented and
	replicable figures and tables. Although good empirical work requires
	that the analyst understand her tools, she must also think about how
	to communicate effectively: ability to reproduce past analyses and
	clean and clear presentations of data summaries are almost as
	important as clear writing in this regard.

	\nobibliography*

	\section{Books}\vspace{-2em}
	I'm am not requiring any particular books this term. The readings will be
	drawn from a variety of sources. I will try to make most of them
	available to you as we go if you can't find them easily online yourselves.

	\subsection{Recommended}
	No book is perfect for all students. I suggest you ask around, look at
	other syllabi online, and just browse the shelves at the library and
	used bookstores to find books that make things clear to you.  I will be
	adding some recommendations here. Let me know now if you have favorites.

	\paragraph{Self-Help}
	If you discover any books or websites that are particularly useful to you, please
	alert me and the rest of the class about them. Thanks!

	\section{Schedule}

	\textbf{Note: } This schedule is preliminary and subject to change. If
	you miss a class make sure you contact me or one of your colleagues to
	find out about changes in the lesson plans or assignments.

	The idea behind the sequencing here is to start as simple as possible
	and complicate later.

	\textbf{Data: } I'll be bringing in data that I have on hand. This
	means our units of analysis will often be individual people or perhaps
	political or geographic units, mostly in the United States. I'd love
	to use other data, so feel free to suggest and provide it to me ---
	come to office hours and we can talk about how to use your favorite
	datasets in the class.

	\textbf{Theory: } This class is about description, estimation, comparisons,
	and inference. Yet, statistics as a discipline exists to help us
	understand more than why the linear model works as it does. Thus,
	social science theory cannot be far from our minds as we think about
	what makes a given data analytic strategy meaningful. That is, while
	we spend a term thinking a lot about how to make meaningful statements
	about statistical inference, we must also keep substantive
	significance foremost in our minds.


\end{introstuff}


\SetDate[23/08/2015]

\section{0---\themonth~\the\day--- Overview, Statistics, Data, Variables}

Introductions.

Introduction to the class and my thoughts about statistics. Your thoughts
about statistics. Your thoughts about learning difficult topics?

To start Daily R we will need datasets. We need to talk about which datasets
might be of interest to the class and get them online by the end of Wednesday.

What is the point of statistics? What is data? What is a variable

\part{Description}


\AdvanceDate[7]
\section{1---\themonth~\the\day---Description in One Dimension}

What makes a description useful or not useful? What is a good description? How
would we know whether we have a good one or a bad one? Are there descriptions
that are particularly robust to observations we'd like to ignore or mostly
ignore because they are apt to mislead us?

\subsection{Reminder:} Bring laptops if you have them. Those bringing
laptops should have R installed.

\subsection{Read:}
\textbf{Henceforth, ``*'' means ``recommended'' or ``other useful'' reading. The readings not marked with ``*'' are required.}

*\citealp[Chap 1--3]{wilcox2012introduction}

*\citealp[Chap 2--3]{kaplan2012ism}


\AdvanceDate[7]
\section{2---\themonth~\the\day---Description in Two Dimensions I}

\subsection{Topics:}  Linear data models and fitting.

\subsection{Questions:} Why use straight lines to describe relationships? On
what basis should we choose a straight line (why choose one straight line over
others)? How to interpret slopes and intercepts (i.e. the descriptors of a
line) given different ways of choosing lines? What is the relationship between
a straight line and differences between two groups? When might we want to
identify and perhaps diminish the influence of particular individual
observations on an overall linear description?

\subsection{Read:}

*\citealp[Chap 2]{achen82}

*\citealp[Chap 4,5--8]{kaplan2012ism}

*\citealp[Chap 2--3]{james2013introduction}

*\citealp[Chap 1]{berk2008statistical}

*\citealp{berk2010you} (especially level 1 regression)

*\citealp[Chap 5 and 10]{wilcox2012introduction}

\AdvanceDate[7]
\section{3---\themonth~\the\day---Description in Two Dimensions II}

\subsection{Topics:} Nonlinear data models and fitting

\subsection{Questions:} What other criteria for line choice might we have?
Re-interpreting linear description as both smooth and comparison. How to argue
that you have smoothed the data appropriately? How does the use of dummy
variables or indicator variables and/or interaction terms allow us to engage
with theoretical expectations that are not linear? What about transformations
of predictors like simple polynomials? Or piecewise fits (smooth ones using
splines, or not smooth ones using indicators and interactions)? Or locally
smooth fits?

\subsection{Read:}
*\citealp[Chap 3,7]{james2013introduction}\footnote{See
	\url{http://www-bcf.usc.edu/~gareth/ISL/} for materials}


\part{Creating Interpretable Comparisons}

\AdvanceDate[7]
\section{4---\themonth~\the\day---Conditional description and Causal Counterfactuals}

\subsection{Topics:} Counterfactual approaches to formalizing causal
statements; potential outcomes
\subsection{Question:} What do we mean when we say "X caused Y"? What is the
counterfactual interpretation of this statement?

\subsection{Read:}

Brady on causality.
?SOme introduction to potential outcomes.


\AdvanceDate[7]
\section{5---\themonth~\the\day---Experiments}

\subsection{Topics}
\subsection{Questions}
\subsection{Reading}
Kinder and Palfrey on experiments
\AdvanceDate[7]
\section{6---\themonth~\the\day---Statistical Adjustment I}

\subsection{Topics:} Stratification

\AdvanceDate[7]
\section{7---\themonth~\the\day---Statistical Adjustment II}

\subsection{Topics:} Covariance adjustment, adjustment by linear model
(interaction terms for stratification, or direct covariance adjustment aka
"controlling for")

\AdvanceDate[7]
\section{8---\themonth~\the\day---No Class}


\AdvanceDate[7]
\section{9--\themonth~\the\day--Statistical Adjustment III}

\subsection{Topics:} Matching methods
\subsection{Read:}
\subsection{Do:}


\AdvanceDate[7]
\section{10---\themonth~\the\day---Statistical Adjustment IV}

\subsection{Topic:} Difference in differences; regression discontinuity;
instrumental variables
\subsection{Read:}

\part{Estimation and Inference}

\AdvanceDate[7]

\section{11---\themonth~\the\day---Estimation I}

\subsection{Topics:} What are we estimating? Why? How?
\subsection{Read:}

\AdvanceDate[7]
\section{12---\themonth~\the\day---Estimation II}

\subsection{Topics:} Evaluating bias and consistency

\AdvanceDate[7]
\section{\themonth~\the\day---No Class}

\AdvanceDate[7]
\section{13---\themonth~\the\day---Class Choice}
\subsection{Topics:}
\subsection{Read:}
\subsection{Do:}

\AdvanceDate[7]
\section{14---\themonth~\the\day---Class Choice, Optional Extra Class}
\subsection{Topics:}
\subsection{Read:}
\subsection{Do:}

\part{References}
%\bibliography{/Users/jwbowers/Documents/PROJECTS/BIB/big}
\bibliography{syllabus}

\end{document}



